\documentclass[12pt]{article}

\usepackage[round]{natbib}
\usepackage{hyperref}
\hypersetup{colorlinks=true,
    linkcolor=blue,
    citecolor=blue,
    filecolor=blue,
    urlcolor=blue,
    unicode=false}
\urlstyle{same}

\usepackage{enumitem,amssymb}
\newlist{todolist}{itemize}{2}
\setlist[todolist]{label=$\square$}
\usepackage{pifont}
\newcommand{\cmark}{\ding{51}}%
\newcommand{\xmark}{\ding{55}}%
\newcommand{\done}{\rlap{$\square$}{\raisebox{2pt}{\large\hspace{1pt}\cmark}}%
\hspace{-2.5pt}}
\newcommand{\wontfix}{\rlap{$\square$}{\large\hspace{1pt}\xmark}}

\begin{document}

\title{HA Checklist}
\author{Spencer Smith}
\date{\today}

\maketitle

% Show an item is done by   \item[\done] Frame the problem
% Show an item will not be fixed by   \item[\wontfix] profit

This checklist is intended to help with a high-quality HA document. The HA will
help you look at your requirements differently.  There is a good chance that you
will discover new requirements as you complete the HA exercise.

\begin{itemize}

\item Follows writing checklist (full checklist provided in a separate document)
  \begin{todolist}
  \item \LaTeX{} points
  \item Structure
  \item Spelling, grammar, attention to detail
  \item Avoid
  \href{https://www.brafton.com/blog/content-writing/anti-fluff-content-writing/}
  {low information content phrases} (like replacing ``in order to'' with ``to'')
  \item Writing style
  \end{todolist}

\item Follows the template, all parts present
  \begin{todolist}
  \item All parts are present
  \item If there are any changes, the changes are described in the introduction
  \end{todolist}

\item HA checklist
  \begin{todolist}
  \item Each row in the table includes information on how the failure will be
  mitigated
  \item There is traceability between the failure cases and the associated
  mitigation requirements
  \item You have thought broadly about safety. For instance, security, the
  psychological impacts of the software, and other harms like that should be
  considered. As an example, there is considerable  discussion right now about
  the addictiveness of software products or the adverse effects it can have on
  well-being, including self-image, etc. These should be considered especially
  for games and other interactive applications being used by the general public.
  \item The recommended actions should be clear.  As an example, saying
  something like "Better secure user data" is not an actionable without more
  detail. It should still be abstract, since this is still the requirement
  phase, but that is too unspecific.
  \end{todolist}

\item Avenue Rubric
  \begin{todolist}
  \item You have checked your work against the grading rubric on Avenue
  \end{todolist}

\end{itemize}

\bibliographystyle {plainnat}
\bibliography{../../refs/References}

\end{document}