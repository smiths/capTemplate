\documentclass[12pt]{article}

\usepackage{enumitem,amssymb}
\newlist{todolist}{itemize}{2}
\setlist[todolist]{label=$\square$}
\usepackage{pifont}
\newcommand{\cmark}{\ding{51}}%
\newcommand{\xmark}{\ding{55}}%
\newcommand{\done}{\rlap{$\square$}{\raisebox{2pt}{\large\hspace{1pt}\cmark}}%
\hspace{-2.5pt}}
\newcommand{\wontfix}{\rlap{$\square$}{\large\hspace{1pt}\xmark}}

\usepackage{hyperref}
\hypersetup{colorlinks=true,
    linkcolor=blue,
    citecolor=blue,
    filecolor=blue,
    urlcolor=blue,
    unicode=false}
\urlstyle{same}

\begin{document}

\title{System Verification and Validation Plan Checklist}
\author{Spencer Smith}
\date{\today}

\maketitle

% Show an item is done by   \item[\done] Frame the problem
% Show an item will not be fixed by   \item[\wontfix] profit

\begin{itemize}

\item Follows writing checklist (full checklist provided in a separate document)
  \begin{todolist}
  \item \LaTeX{} points
  \item Structure
  \item Spelling, grammar, attention to detail
  \item Avoid low information content phrases
  \item Writing style
  \end{todolist}

\item Follows the template, all parts present
  \begin{todolist}
  \item Table of contents
  \item Pages are numbered
  \item Revision history included for major revisions
  \item Sections from template are all present
  \item Symbolic constants are used rather than ``magic'' numbers.  Symbolic
    constants are used to improve maintainability and to increase
    understandability
  \item Specific values are provided for all symbolic constants
  \end{todolist}

\item Grammar, spelling, presentation
  \begin{todolist}
  \item No spelling mistakes (use a spell checker!)
  \item No grammar mistakes (review, ask someone else to review (at least a few
    sections))
  \item Paragraphs are structured well (clear topic sentence, cohesive)
  \item Paragraphs are concise (not wordy)
  \item No Low Information Content (LIC) phrases
    (\href{https://www.webpages.uidaho.edu/range357/extra-refs/empty-words.htm}{List
      of LIC phrases})
  \item All hyperlinks work
  \item Every figure has a caption
  \item Every table has a heading
  \item Symbolic names are used for quantities, rather than literal values
  \end{todolist}

\item LaTeX
  \begin{todolist}
  \item Template comments do not show in the pdf version, either by
    removing them, or by turning them off.
  \item References and labels are used so that maintenance is feasible
\end{todolist}

\item Overall qualities of documentation
  \begin{todolist}
\item Test cases include SPECIFIC input
\item Test cases include EXPLICIT output
\item Description over specification, when appropriate
\item Plans for what to do with description data (performance, usability, etc).
  This may involve saying what plots will be generated.
\item Plans to quantify error for scalar values using relative error
\item Plans to quantify error for vector and matrix values using a norm of an error
  vector (matrix)
\item Plans are feasible (can be accomplished with resources available)
\item Plans are ambitious enough for an A+ effort
\item Survey questions for usability survey are in an Appendix (if appropriate)
\item Specific plans for task based inspection, if appropriate (not just saying
inspection will be done, but details on how)
\item Provided adequate detail on non-dynamic testing.  Statements liks ``We
will perform a code walkthrough with our stakeholders'' are accompanied by
details, such as a checklist of items to go through during a walkthrough.
\item Very careful use of random testing
\item Specific programming language is listed
\item Specific linter tool is listed (if appropriate)
\item Specific coding standard is given
\item Specific unit testing framework is given
\item Investigation of code coverage measuring tools
\item Specific plans for Continuous Integration (CI), or an explanation that CI
  is not being done and why not
\item Specific performance measuring tools listed (like Valgrind), if
  appropriate
\item If you are referencing an outside standard like the Web Content
Accessibility Guidelines
(\href{https://www.w3.org/WAI/standards-guidelines/wcag/} {WCAG}), refer back to
it when talking about it. Don't just say ``perform WCAG checks to validate
accessibility'' -- say what tests you are planning on performing. If they are
provided by WCAG, reference the specific tests you'd like to use.
\item Traceability between test cases and requirements is summarized (likely in
  a table).  The traceability matrix shows a test case for each requirement, or
  a non-dynamic technique is used for that requirement.
\item 
\end{todolist}

\item Avenue rubric
  \begin{todolist}
  \item More than 5 peer review issues created for another team
  \item Should have enough redundancy in testing.  Ideally there should be more
  than one approach for verification for each requirement.
  \item Extras should be clearly identified and should be feasible.  The TA
  should have enough information to be able to provide feedback.
  \item A case should be made for why the extras will improve the project, and
  thus prove that they are not an afterthought.
\end{todolist}

\end{itemize}

\end{document}
