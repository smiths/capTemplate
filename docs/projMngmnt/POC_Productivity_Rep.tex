\documentclass{article}

\usepackage{float}
\restylefloat{table}

\usepackage{booktabs}

\title{Team Productivity: POC\\\progname}

\author{\authname}

\date{}

\input{../Comments.text}
%% Common Parts

\newcommand{\progname}{Software Engineering} % PUT YOUR PROGRAM NAME HERE
\newcommand{\authname}{Team \#12, Lower Earth Orbiters
\\ Buu Ha
\\ Rishi Vaya
\\ Diamond Ahuja
\\ Dhruv Cheemakurti
\\ Umang Rajkarnikar } % AUTHOR NAMES                  

\usepackage{hyperref}
    \hypersetup{colorlinks=true, linkcolor=blue, citecolor=blue, filecolor=blue,
                urlcolor=blue, unicode=false}
    \urlstyle{same}
                                

\begin{document}

\maketitle

This document summarizes the contributions of each team member up to the POC
Demo.  The time period of interest is the time between the beginning of the term
and the POC demo.

\wss{Please do not delete sections or modify the table formats.  Standardization
helps the instructors and TAs to review these documents. You may add sections to
the report.}

\wss{Remove the instructor guidelines/comments in the final version.}

\wss{Fill in the requested information, like project url, TA name, supervisor name, etc.}

\section{Demo Plans}

\wss{What is your feasibility risk?}

\wss{What will you be demonstrating}

\section{Team Meeting Attendance}

\wss{For each team member how many team meetings have they attended over the
time period of interest.  This number should be determined from the meeting
issues in the team's repo.  The first entry in the table should be the total
number of team meetings held by the team.}

\begin{table}[H]
\centering
\begin{tabular}{ll}
\toprule
\textbf{Student} & \textbf{Meetings}\\
\midrule
Total & Num\\
Name 1 & Num\\
Name 2 & Num\\
Name 3 & Num\\
Name 4 & Num\\
Name 5 & Num\\
\bottomrule
\end{tabular}
\end{table}

\wss{If needed, an explanation for the counts can be provided here.}

\section{Supervisor/Stakeholder Meeting Attendance}

\wss{For each team member how many supervisor/stakeholder team meetings have
they attended over the time period of interest.  This number should be determined
from the supervisor meeting issues in the team's repo.  The first entry in the
table should be the total number of supervisor and team meetings held by the
team.  If there is no supervisor, there will usually be meetings with
stakeholders (potential users) that can serve a similar purpose.}

\noindent \textbf{Supervisor's Name: } [fill in this information]

\begin{table}[H]
\centering
\begin{tabular}{ll}
\toprule
\textbf{Student} & \textbf{Meetings}\\
\midrule
Total & Num\\
Name 1 & Num\\
Name 2 & Num\\
Name 3 & Num\\
Name 4 & Num\\
Name 5 & Num\\
\bottomrule
\end{tabular}
\end{table}

\wss{If needed, an explanation for the counts can be provided here.}

\section{Lecture Attendance}

\wss{For each team member how many lectures have they attended over the time
period of interest.  This number should be determined from the lecture issues in
the team's repo. You can find the number of lectures in the time period of
interest by looking at the
\href{https://calendar.google.com/calendar/u/0/embed?src=rnboqiaki1k2la7rpu3bn0um58@group.calendar.google.com&ctz=America/Toronto}
{Google calendar} for the capstone course. The first entry in the
table should be the total number of lectures held for the class during the period of interest.}

\wss{NOTE: There will be approximately 13 lectures between the start of class
and the POC demos}

\begin{table}[H]
\centering
\begin{tabular}{ll}
\toprule
\textbf{Student} & \textbf{Lectures}\\
\midrule
Total & Num\\
Name 1 & Num\\
Name 2 & Num\\
Name 3 & Num\\
Name 4 & Num\\
Name 5 & Num\\
\bottomrule
\end{tabular}
\end{table}

\wss{If needed, an explanation for the lecture attendance can be provided here.}

\section{TA Document Discussion Attendance}

\wss{For each team member how many of the informal document discussion meetings
with the TA were attended over the time period of interest. The first row is the
total number of discussion meetings scheduled by the TA with the team.}

\noindent \textbf{TA's Name: } [fill in this information]

\begin{table}[H]
\centering
\begin{tabular}{ll}
\toprule
\textbf{Student} & \textbf{Meetings}\\
\midrule
Total & Num\\
Name 1 & Num\\
Name 2 & Num\\
Name 3 & Num\\
Name 4 & Num\\
Name 5 & Num\\
\bottomrule
\end{tabular}
\end{table}

\wss{If needed, an explanation for the attendance can be provided here.}

\section{Commits}

\wss{For each team member how many commits to the main branch have been made
over the time period of interest.  The total is the total number of commits for
the entire team since the beginning of the term.  The percentage is the
percentage of the total commits made by each team member.}

\wss{There are also columns for lines added and lines deleted to give a sense of
the magnitude of each team member's contributions.  Note that these numbers can
be somewhat misleading, as a team member who does refactoring may have high
numbers in both columns, while a team member who implements a large new feature
may have a high number of lines added, but few lines deleted.  These numbers
will be interpreted with care.}

\begin{table}[H]
\centering
\begin{tabular}{lllll}
\toprule
\textbf{Student} & \textbf{Commits} & \textbf{Percent} & \textbf{Lines Added} &
\textbf{Lines Deleted}\\
\midrule
Total & Num & 100\% & total added & total deleted\\
Name 1 & Num & \% & num & num\\
Name 2 & Num & \% & num & num\\
Name 3 & Num & \% & num & num\\
Name 4 & Num & \% & num & num\\
Name 5 & Num & \% & num & num\\
\bottomrule
\end{tabular}
\end{table}

\wss{If needed, an explanation for the counts can be provided here.  For
instance, if a team member has more commits to unmerged branches, these numbers
can be provided here.  If multiple people contribute to a commit, git allows for
multi-author commits.}

\section{Issue Tracker}

\wss{For each team member how many issues have they authored (including open and
closed issues (O+C)) and how many have they been assigned (only counting closed
issues (C only)) over the time period of interest.}

\begin{table}[H]
\centering
\begin{tabular}{lll}
\toprule
\textbf{Student} & \textbf{Authored (O+C)} & \textbf{Assigned (C only)}\\
\midrule
Name 1 & Num & Num \\
Name 2 & Num & Num \\
Name 3 & Num & Num \\
Name 4 & Num & Num \\
Name 5 & Num & Num \\
\bottomrule
\end{tabular}
\end{table}

\wss{If needed, an explanation for the counts can be provided here.}

\section{CICD via GitHub Actions}

\wss{Say how CICD technology is used and will be used in your project.}

\wss{Provide links to your CICD yaml files}

\section{Extras}

\wss{What is the plan (as documented in TeamComposition.csv) for the team's
extras?  Should the extras be modified now that the team knows more about the
project?}

\section{Team Charter Trigger Items}

\wss{Provide a summary of the quantified triggers identified in the team's
charter.}

\wss{Provide a list of any violations of the triggers.  If the team wishes, the
violations can be summarized on aggregate, instead of naming specific team
members.}

\wss{Provide a plan to address the violations.  This could include revising the
triggers, if they are found to be too weak, strong or ambiguous.}

\section{Additional Productivity Metrics}

\wss{If your team has additional metrics of productivity, please feel free to
add them to this report.  If not, please explicitly state that there are no additional
metrics.}

\wss{Additional metrics can include things like code reviews done, pull requests
created, count of joining meetings late, count of number of times contributions
had to be corrected, number of internal deadlines missed, test cases written, etc.}

\wss{We are looking for data on these metrics, not just a list of additional
metrics the team is planning on tracking.  However, if all you have is a plan,
please share it here.}

\end{document}